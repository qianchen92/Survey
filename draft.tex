%\begin{section}{One-time linearly homomorphic signature based on the Ring-SIS problem}
% 
%This idea comes from the one-time signature based on the Ring-SIS problem proposed by Lyubashevsky and Micciancio~\cite{DBLP:conf/tcc/LyubashevskyM08}.
% 
%\begin{construction}{$R$-SIS based one-time linearly homomorphic signature}
% 
%  \begin{description}
%  \item[$\Setup(1^\lambda) \to (\SSK, \SVK)$] :
%    \begin{enumerate}
%    \item We specify the ring-SIS problem for the degree-$n$ ring $R$ over $\mathbb{Z}$. The security of the one-time signature scheme is based on the $R$-SIS problem.
%    \item Take $\ell \approx log(q)$. We draw a random vector $\vec{a} \sample R_q^{\ell}$.
%    \item Then we draw a matrix $X$ composed by small coefficients in the ring $R_q$.
%    \item Compute $f_{\vec{a}}(X) = \vec{a}^{T} \cdot X$.
%    \end{enumerate}
%  \end{description}
%\end{construction}
% 
%\end{section}
\begin{section}{Fully homomorphic encryption scheme}
  In this section we present several fully homomorphic encryption scheme.

  \begin{subsection}{Third generation of fully homomoprhic encryption scheme}

    Here we present the third generation of fully homomorphic encryption scheme~\cite{DBLP:conf/crypto/GentrySW13}.
    
    At first, we discribe the basic encryption scheme without the homomorphic encryption.

    \begin{description}
    \item[$\boldmath{\Setup(1^\lambda, 1^L)}$]:
      \begin{enumerate}
      \item a
      \end{enumerate}
      
    \end{description}
  \end{subsection}

\end{section}

\begin{section}{Designed Verifier Non-Interactive proof system}

  Our scheme based on the $\Sigma$-protocol proposed by Baum~\etal~\cite{DBLP:journals/iacr/BaumDOP16} and we instantiate the transformation from $\Sigma$-protocol to $NIZK$ without random oracle~\cite{DBLP:conf/pkc/ChaidosG15}.


  \begin{subsection}{$\Sigma$-protocol for encryption of $0$ or $1$}
    In this subsection, we will try to instantiate the $\Sigma$-protocol for encryption of $0$ or $1$ with the lattice assumptions.
    And try to proof the soundness with unique identifiable challenge property for a such $\Sigma$-protocol.
    
    

    \begin{description}
      \item[$\Prove((ek,c), (m,r), (m, r))$]:
        
    \end{description}


  \end{subsection}


  \begin{subsection}{$DVNIZK$ for the correct committed value}
    
    In this section we will give an instantiation of the sigma protocol  for encryption of $0$ or $1$ by Regev's encryption scheme.

    \begin{description}
    
    \item[$\Setup(1^\lambda)$]:
      
      \begin{enumerate}
      \item We draw a random matrix $A \in R_q^{n\times m}$ and a vector $\vec{s} \in R_q^{m \times 1}$.
      \end{enumerate}
    \end{description}
  \end{subsection}

\end{section}