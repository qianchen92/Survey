%\begin{section}{One-time linearly homomorphic signature based on the Ring-SIS problem}
% 
%This idea comes from the one-time signature based on the Ring-SIS problem proposed by Lyubashevsky and Micciancio~\cite{DBLP:conf/tcc/LyubashevskyM08}.
% 
%\begin{construction}{$R$-SIS based one-time linearly homomorphic signature}
% 
%  \begin{description}
%  \item[$\Setup(1^\lambda) \to (\SSK, \SVK)$] :
%    \begin{enumerate}
%    \item We specify the ring-SIS problem for the degree-$n$ ring $R$ over $\mathbb{Z}$. The security of the one-time signature scheme is based on the $R$-SIS problem.
%    \item Take $\ell \approx log(q)$. We draw a random vector $\vec{a} \sample R_q^{\ell}$.
%    \item Then we draw a matrix $X$ composed by small coefficients in the ring $R_q$.
%    \item Compute $f_{\vec{a}}(X) = \vec{a}^{T} \cdot X$.
%    \end{enumerate}
%  \end{description}
%\end{construction}
% 
%\end{section}
\begin{section}{Preliminaries}
%  \begin{definition}{$\Sigma$-Protocol}
%    \todo{complete the definition}
%  \end{definition}
% 
% 
%  \begin{definition}{\textsf(Soundness with unique identifiable challenge)}
%    A sigma protocol $(\Prove_{\Sigma}, \Verif_{\Sigma})$ has a unique identifiable challenge \wrt an NP-relation $\mathcal{R}_{guilt}$, if there is a PPT algorithm $\mathcal{E}$ which takes as the input the statement, the witness and initial message and returns the unique challenge $e$ can be answered.
% 
%    Formally we can define this notion as follows:
%    
%    For all $x, w_{guilt}, a, e, z$ that $(x, w_{guilt}) \in \mathcal{R}_{guilt}$ and $\Verif_{\Sigma}(x, a, e, z) = \True$, the PPT algorithm $\mathcal{E}$ verifies $e = \mathcal{E}(x, w_{guilt}, a)$.
%    
%  \end{definition}
%\end{section}
% 
%\begin{section}{Designed Verifier Non-Interactive proof system}
% 
%  Our scheme based on the $\Sigma$-protocol proposed by Baum~\etal~\cite{DBLP:journals/iacr/BaumDOP16} and we instantiate the transformation from $\Sigma$-protocol to $DV-NIZK$ without random oracle~\cite{DBLP:conf/pkc/ChaidosG15}.
% 
%  \begin{subsection}{$\Sigma$-protocol for encryption of $0$ or $1$}
%    In this subsection, we instantiate the $\Sigma$-protocol for encryption of $0$ or $1$ with the lattice assumptions.
%    And prove the soundness with unique identifiable challenge property for a such $\Sigma$-protocol.
%    
%    Here we instantiate with the GSW~\cite{DBLP:conf/crypto/GentrySW13} fully homomorphic encryption scheme with adaptations.
%    \begin{description}
% 
%      \item[$\Setup(1^\lambda, 1^L)$]:
%        \begin{enumerate}
%        \item Choose a modulus $q$ prime of $\kappa = \kappa(\lambda, L)$ bits, the lattice dimension parameter $n = n (\lambda, L)$ and the error distribution $\chi = \chi(\lambda, L)$ to ensure at least $\lambda$ bit security against the LWE problem.
%        \item Then we compute $m = m (\lambda, L) = O (n~log~q)$, $\ell = \floor{q} + 1$ and $N = (n+1) \times \ell$
%        \item Sample $\vec{t} \sample \mathbb{Z}_q^m$.
%        \item Let $\vec{s} = (1, -t_1, \dots, -t_n) \in \mathbb{Z}_q^{n+1}$.
%        \item Sample a uniformly random matrix $B  \gets \mathbb{Z}_q^{m\times n}$ and a noise vector $\vec{e} \gets \chi^{m}$.
%        \item Compute $\vec{b} = B \cdot \vec{t} + \vec{e}$. 
%        \item Set the matrix $A = [\vec{b} | B ] \in \mathbb{Z}_q^{m \times (n+1)}$, remark that $A \cdot \vec{s} = \vec{e}$.
%        \item Output $PP = (n, q, \chi, m)$, $\SK = \vec{s} \in \mathbb{Z}_q^{n+1}$ and $\PK = A \in \mathbb{Z}_q^{m \times (n+1)}$.
%        \end{enumerate}
%      \item[$\Com((\PK, C), (\mu, R))$]:
% 
%        We begin with the specification of each argument:
%        \begin{enumerate}
%        \item $\PK = A \in\mathbb{Z}_q^{m \times (n+1)}$,
%        \item $C = \{0,1\}^{N \times N}$.
%        \item $\mu \in \mathbb{Z}_q$.
%        \item $R \in \{0,1\}^{N \times m}$
%        \end{enumerate}
%        
%        The commitment algorithm proceeds as follows:
% 
%        \begin{enumerate}
%        \item Choose $m_a \gets \{1\}||\{0,1\}^{l-2} \cap \mathbb{Z}_q$, $R_a \gets \{0,1\}^{N \times m}$ and $R_b \gets \{0,1\}^{N \times m}$.
%        \item Compute
%          \begin{align*}
%            C_a  &= \Flatten(m_a \cdot Id_{N} + \BitDecomp(R_a \cdot A)) \in \mathbb{Z}_q^{N \times N}, \\
%            C_b  &= \Flatten(-\mu \cdot m_a \cdot Id_{N} + \BitDecomp(R_b \cdot A)) \in \mathbb{Z}_q^{N \times N}.
%          \end{align*}
%        \end{enumerate}
% 
%     \item[$\Prove(\PK, C, \mu, R, f, m_a, R_a, m_b, R_b, e \in \mathbb{Z}_q)$]:
%        \begin{enumerate}
%        \item Compute 
%          \begin{align*}
%            f &\gets \Flatten(e \cdot I_N) \cdot  \Flatten(m \cdot I_N) + \Flatten(m_a \cdot I_N) \in \mathbb{Z}_q^{N \times N},\\
%            z_a &\gets \Flatten(e \cdot I_N)\cdot \BitDecomp(R) + \BitDecomp(R_a),\\
%            z_b &= \Flatten((f-e)\cdot I_N) \cdot \BitDecomp(R) + \BitDecomp(R_b).
%          \end{align*}
%        \item Output $(f, z_a, z_b)$.
%        \end{enumerate}
% 
%      \item[$\Verif(\PK, c, C_a, z_a, C_b, z_b, e, f)$]:
%        \begin{enumerate}
%        \item Verify that $(C_a, C_b) \in (\{0,1\}^{(l-1) \times N})^2$.
%        \item Verify that $f \in \mathbb{Z}_q$.
%        \item Verify that $(z_a, z_b) \in  (\{0,1\}^{(l-1) \times m)})$.
%        \item Verify that
%          \begin{align*}
%            {\mathsf{MultConst}}(C, e) &= FHE.\Enc(f; z_a) & {\mathsf{MultConst}}(C, f-e) + C_b &= FHE.\Enc(0; z_b)
%          \end{align*}
% 
%          These two equations can be written in the following forms.
% 
%          \begin{align*}
%            \Flatten(f \cdot I_N) &= \Flatten(\Flatten(e \cdot I_N) \cdot \Flatten(\mu \cdot I_N) \\
%                                  &+ \Flatten(m_a \cdot I_N))\\
%            0 &= \Flatten(\Flatten((f-e) \cdot I_N) + \Flatten(m_b \cdot I_N))
%          \end{align*}
%        \end{enumerate}
% 
% 
% 
% 
%        \begin{lemma}
%          The previous described $\Sigma$-protocol is sound with unique identifiable challenge for the $\mathcal{R}_{guilt} = \{(x = (\PK, c),w_{guilt} = (m, R)) | c \in \mathcal{C}~and~\Dec(\SK, c) \not \in \{ 0, 1 \}~and~{\mathsf{VerifyKey}(1^{\lambda}, \PK, SK)} = \True\}$. 
%        \end{lemma}
% 
%        \begin{proof}
%          As the adversary has responded correctly for the challenge $e$ which means the proof $\pi = (f, z_a, z_b)$ verifies that
%          \begin{align*}
%            \Flatten(f \cdot I_N) &= \Flatten(\Flatten(e \cdot I_N) \cdot \Flatten(m \cdot I_N) \\
%                                  &+ \Flatten(m_a \cdot I_N))\\
%            0 &= \Flatten(\Flatten((f-e) \cdot I_N)\cdot \Flatten(m \cdot I_N) + \Flatten(m_b))
%          \end{align*}
%          
%          By substitute $f$ in th second equation, we have the following equality:
% 
%          \begin{equation}
%            \begin{split}
%              0 =& \Flatten(\Flatten(e \cdot I_N) \cdot \Flatten(m \cdot I_N) \cdot \Flatten(m \cdot I_N) \\
%              &- \Flatten(m_a \cdot I_N) \cdot \Flatten(m \cdot I_N) \\
%              &- \Flatten(e \cdot I_N)\cdot \Flatten(m \cdot I_N) \\
%              &+ \Flatten(m_b \cdot I_N))\\
%            \end{split}
%          \end{equation}
%          
%          Using the lemma~\ref{rm-flatten}. We can multiply the both sides by the vector $(1, 2, \dots, 2^{N-1})^T$. We obtain the following equation.
%          \begin{align*}
%            (e\cdot m^2 - m_a \cdot m - e \cdot m + m_b) \cdot \left( \begin{smallmatrix} 1\\ 2\\ \vdots \\ 2^{N-1} \end{smallmatrix}\right) &= \left( \begin{smallmatrix} 0\\ 0\\ \vdots \\ 0 \end{smallmatrix}\right)~mod~q\\
%            e \cdot m^2 - m_a \cdot m - e \cdot m + m_b &= 0 mod q
%          \end{align*}
%          
% 
%          As we have $m \not \in \{0,1\}$ and $m \in \mathbb{Z}_q^*$, thus $e$ is uniquely defined by $\frac{m_a \cdot m + m_b}{(m -1 ) \cdot m}$. 
%        \end{proof}
%        
%        
%    \end{description}
% 
% 
%  \end{subsection}

  \begin{subsection}{$\Sigma$-protocol revisited by using simpler Gadget matrix presentation}

    This $\Sigma$-protocol can prove the ciphertext of $0$ or $1$.

    \begin{description}
    \item[$\KeyGen(1^\lambda, 1^L)$]:
      \begin{enumerate}
      \item Choose a modulus $q$ prime of $\kappa = \kappa(\lambda, L)$ bits, the lattice dimension parameter $n = n(\lambda, L)$ and the error distribution $\chi = \chi(\lambda, L)$ to ensure at least $\lambda$ bits security against the LWE problem.
      \item Compute $m = m(\lambda, L) = O(n~log~q)$, $\ell = \floor{q}+1$ and $N = (n+1) \cdot \ell$.
      \item Sample $\vec{t} \sample \mathbb{Z}_q^m$ and set $\vec{s} = (1, -t_1, \dots, -t_n) \in \mathbb{Z}_q^{n+1}$.
      \item Sample a uniformaly random matrix $B \gets \mathbb{Z}_q^{m \times n}$ and a noise vector $\vec{e} \gets \chi^m$.
      \item Compute $\vec{b} = B \cdot \vec{t} + \vec{e}$.
      \item Set the matrix $A = [\vec{b}|B] \in \mathbb{Z}_q^{m \times (n+1)}$, remark that $A \cdot \vec{s} = \vec{e}$.
      \item Output $\PPP = (n, q, \chi, m)$, $\SK = \vec{s} \in \mathbb{Z}_q^{n+1}$ and $\PK = A \in \mathbb{Z}_q^{m \times (n+1)}$.
      \end{enumerate}
    \item[$\Com((\PK, C), (\mu, R))$]:
      \begin{enumerate}
      \item Choose $m_a \gets \{1\}||\{0,1\}^{\ell-2} \cap \mathbb{Z}_q$, $R_a \gets \{0,1\}^{N \times m}$ and $R_b \gets \{0,1\}^{N \times m}$.
      \item Compute
        \begin{align*}
          C_a &= A\cdot R_a + m_a \cdot G \in \mathbb{Z}_q^{N \times N}\\
          C_b &= A\cdot R_b - \mu \cdot m_a \cdot G \in \mathbb{Z}_q^{N \times N}          
        \end{align*}
      \item Ouput $C_a, C_b$.
      \end{enumerate}

    \item[$\Prove(\PK, C, m, R, f, m_a, R_a, m_b, R_b, e \in \mathbb{Z}_q)$]:
      \begin{enumerate}
      \item Compute $f = em +m_a$.
      \item Compute $Z_a = R \cdot G^{-1}(e \cdot G) + R_a$.
      \item Compute $Z_b = R \cdot G^{-1}((f-e) \cdot G) + R_b$.
      \item Output $(f, Z_a, Z_b)$.
      \end{enumerate}
      
    \item[$\Verif(\PK, c, C_a, Z_a, C_b, Z_b, e, f)$]:
      \begin{enumerate}
      \item Verify that $(C_a, C_b, f, Z_a, Z_b) \in (\{0,1\}^{(\ell-1) \cdot N})^2 \times \mathbb{Z}_q \times (\{0,1\}^{(\ell-1)\cdot m})^2$.
      \item Verify that
        \begin{align*}
          {\mathsf{MultConst}}(C, e)+C_a &= FHE.\Enc(f; Z_a) & {\mathsf{MultConst}}(C, f-e) + C_b &= FHE.\Enc(0; Z_b)
        \end{align*}
      \end{enumerate}
    \end{description}
  \end{subsection}


  \begin{subsection}{Matrix Fully Homomorphic Encryption~\cite{DBLP:conf/pkc/HiromasaAO15}}
    \begin{description}
    \item[$\KeyGen (1^\lambda, r)$]:
      \begin{enumerate}
      \item Set the parameters $n, q, m, \ell, N$, and $\chi$ a Gaussian distribution.
      \item Set a uniformly random matrix $\mat{A} \sample \mathbb{Z}_q^{n \times m}$ and $\mat{S'} \sample \chi^{r \times n}$.
      \item Choose a noise matrix $\mat{E} \sample \chi^{r \times m}$.
      \item Let $\mat{S}:= [\mat{I}_r || - \mat{S}'] \in \mathbb{Z}_q^{r \times (n+r)}$.
      \item Set the matrix:
        \begin{align*}
          \mat{B} &:= (\frac{\mat{S}'\mat{A} + \mat{E}}{\mat{A}}) \in \mathbb{Z}_q^{(n+r) \times m}
        \end{align*}
      \item Denote $\mat{E}_{i,j} \in \{0, 1\}^{r\times r}$ with $(i,j) \in \{1, \dots ,r\}^2$.
      \item Sample $\mat{R}_{i,j} \sample \{0,1\}^{m \times N}$ and
        \begin{align*}
          \mat{P}_{i,j} &:= \mat{B} \mat{R}_{i,j} + (\frac{\mat{M}_{i,j}\mat{S}}{\mat{0}}) \mat{G} \in \mathbb{Z}_q^{(n+r) \times N}.
        \end{align*}
      \item Output $\PK := (\{\mat{P}_{i,j \in [r]}, \mat{B}\})$ and $\SK = \mat{S}$.
      \end{enumerate}

    \item[$\Enc(\PK, \mat{M} \in \{0,1\}^{r \times r})$] :
      \begin{enumerate}
      \item Sampl a random matrix $\mat{R} \sample \{0,1\}^{m \times N}$.
      \item Compute the ciphertext:
        \begin{align*}
          \mat{C} &:= \mat{B} \mat{R} + \sum_{i,j \in [r] : \mat{M}[i,j] = 1} \mat{P}_{(i,j)} \in \mathbb{Z}_q^{(n+r) \times N}
        \end{align*}
      \end{enumerate}

    \item[$\Dec(\mat{C}, \SK)$]:
      \begin{enumerate}
      \item Compute
        \begin{align*}
          \mat{M} &=(\round{\langle \vec{s}_i, \vec{c}_{j\ell-1} \rangle}_2)_{i,j \in [r]} \in \{0,1\}^{r \times r}
        \end{align*}
      \end{enumerate}
    \end{description}
  \end{subsection}
  
  
  \begin{subsection}{Construction of the DV-NIZK of a encryption of $0$ or $1$}
    
    \begin{description}
    \item[$\KeyGen(1^\lambda, 1^L)$]:
      \begin{enumerate}
      \item Choose a modulus $q$ prime of $\kappa = \kappa(\lambda, L)$ bits, the lattice dimension parameter $n = n(\lambda, L)$ and the error distribution $\chi = \chi(\lambda, L)$ to ensure at least $\lambda$ bits security against the LWE problem.
      \item Compute $m = m(\lambda, L) = O(n~log~q)$, $\ell = \floor{q}+1$ and $N = (n+1) \cdot \ell$.
      \item Sample $\vec{t} \sample \mathbb{Z}_q^m$ and set $\vec{s} = (1, -t_1, \dots, -t_n) \in \mathbb{Z}_q^{n+1}$.
      \item Sample a uniformaly random matrix $B \gets \mathbb{Z}_q^{m \times n}$ and a noise vector $\vec{e} \gets \chi^m$.
      \item Compute $\vec{b} = B \cdot \vec{t} + \vec{e}$.
      \item Set the matrix $A = [\vec{b}|B] \in \mathbb{Z}_q^{m \times (n+1)}$, remark that $A \cdot \vec{s} = \vec{e}$.
      \item Choose a random element $e \in \mathbb{Z}_{q}$.
      \item Compute
        \begin{align*}
          \mat{C}_e &= \mat{A} \mat{R}_e + e \cdot \mat{G}\\
          \mat{C}_E &= \mat{B} \mat{R}_E + 
        \end{align*}
      \item Output $\PPP = (n, q, \chi, m)$, $\SK = \vec{s} \in \mathbb{Z}_q^{n+1}$ and $\PK = A \in \mathbb{Z}_q^{m \times (n+1)}$.
      \end{enumerate}
    \item[$\Prove((\PK, C), (\mu, R))$]:
      \begin{enumerate}
      \item Choose $m_a \gets \{1\}||\{0,1\}^{\ell-2} \cap \mathbb{Z}_q$, $R_a \gets \{0,1\}^{N \times m}$ and $R_b \gets \{0,1\}^{N \times m}$.
      \item Compute
        \begin{align*}
          C_a &= A\cdot R_a + m_a \cdot G \in \mathbb{Z}_q^{N \times N}\\
          C_b &= A\cdot R_b - \mu \cdot m_a \cdot G \in \mathbb{Z}_q^{N \times N}          
        \end{align*}
      \item Compute $f = em +m_a$.
      \item Compute $Z_a = R \cdot G^{-1}(e \cdot G) + R_a$.
      \item Compute $Z_b = R \cdot G^{-1}((f-e) \cdot G) + R_b$.
      \item Output $(f, Z_a, Z_b)$.
      \end{enumerate}
      
    \item[$\Verif(\PK, c, C_a, Z_a, C_b, Z_b, e, f)$]:
      \begin{enumerate}
      \item Verify that $(C_a, C_b, f, Z_a, Z_b) \in (\{0,1\}^{(\ell-1) \cdot N})^2 \times \mathbb{Z}_q \times (\{0,1\}^{(\ell-1)\cdot m})^2$.
      \item Verify that
        \begin{align*}
          {\mathsf{MultConst}}(C, e)+C_a &= FHE.\Enc(f; Z_a) & {\mathsf{MultConst}}(C, f-e) + C_b &= FHE.\Enc(0; Z_b)
        \end{align*}
      \end{enumerate}
    \end{description}
    
    
  \end{subsection}


  \begin{subsection}{More DV-NIZK proofs for the encryption}
    
    \paragraph{Proof for encryption of $0$}
    We have an $C_0$ an encryption of $0$, 
    \begin{align*}
      C_0  = & \Flatten( 0 \cdot I_B + \BitDecomp(R \cdot A))\\
      = & \BitDecomp(R \cdot A)
    \end{align*}

    The prove for the encryption of $0$ is decomposed into two parts.
    \begin{enumerate}
    \item We first prove the fact that $C_0$ is an encryption of $0$ or $1$ using the previous DV-NIZK proof.
    \item Then we prove that $C_0 + C_0$ is also an encryption of $0$ or $1$.
    \end{enumerate}

    \paragraph{Proof for encryption of same message}
    This is straightforward from the previous proof for encryption of $0$. For the ciphertext $C_m^1$ and $C_m^2$, we prove the fact that $C_m^1 - C_m^2$ is an encryption of $0$.

    
  \end{subsection}

  \todo[inline]{A possible way to do the correct the encryption is to use the packing encryption methode presented in~\cite{DBLP:conf/pkc/HiromasaAO15} and instead of publish only $AR+eG$ publish also the encryption of matrix $G^{-1}(eG)$ using the matrix encryption method which allows us to compute $R \cdot G^{-1}(e \cdot G) + R_a$ later}


\end{section}
