%\begin{section}{One-time linearly homomorphic signature based on the Ring-SIS problem}
% 
%This idea comes from the one-time signature based on the Ring-SIS problem proposed by Lyubashevsky and Micciancio~\cite{DBLP:conf/tcc/LyubashevskyM08}.
% 
%\begin{construction}{$R$-SIS based one-time linearly homomorphic signature}
% 
%  \begin{description}
%  \item[$\Setup(1^\lambda) \to (\SSK, \SVK)$] :
%    \begin{enumerate}
%    \item We specify the ring-SIS problem for the degree-$n$ ring $R$ over $\mathbb{Z}$. The security of the one-time signature scheme is based on the $R$-SIS problem.
%    \item Take $\ell \approx log(q)$. We draw a random vector $\vec{a} \sample R_q^{\ell}$.
%    \item Then we draw a matrix $X$ composed by small coefficients in the ring $R_q$.
%    \item Compute $f_{\vec{a}}(X) = \vec{a}^{T} \cdot X$.
%    \end{enumerate}
%  \end{description}
%\end{construction}
% 
%\end{section}